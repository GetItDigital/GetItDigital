\section{Der magnetische Widerstand}
\begin{frame} \ftx{Magnetischer Widerstand}\index{Magnetischer Widerstand}

	\s{
		Im Kapitel \ref{durchflutung} wurde bereits die magnetische Spannung thematisiert, die auch Durchflutung genannt wird, sowie im Kapitel \ref{Flussdichte} der magnetische Fluss, der das Äquivalent zum elektrischen Strom darstellt. Es liegt daher die Vermutung nahe, dass es in einem magnetischen Kreis äquivalent zum ohmschen Widerstand auch einen magnetischen Widerstand\index{Magnetischer Widerstand} gibt. Er hat das Formelzeichen $R_{\mathrm{m}}$ und die Einheit $\frac{\mathrm{A}}{\mathrm{V}\mathrm{s}}$ und wird auch Reluktanz \index{Reluktanz}genannt.

		Ein Beispiel für einen magnetischen Kreis ist in Abbildung \ref{AbbMagnKreis} durch einen einfachen Eisenring mit einer einseitigen Spule dargestellt. Die Spule erzeugt eine Durchflutung $\varTheta$ analog zur elektrischen Spannung. Der Eisenkreis besteht aus vier Teilwiderständen, da der magnetische Widerstand sowohl von der Länge als auch von dem durchflossenen Querschnitt abhängig ist. Die vier Widerstände werden vom magnetischen Fluss $\varPhi$ durchströmt.
	}

	\fu{
		\def\bi{3.0}%Breite innen
\def\hi{2.0}%Höhe innen
\def\ba{5.0}%breite außen	
\def\ha{4.0}%Höhe außen		
\def\dx{0.8}%x-Verschiebung
\def\dy{0.5}%y-Verschiebung		
\def\dr{0.02}% round corner correction		
\def\countPrim{4}%Anzahl Primärwindungen

%Berechnungen
\def\lx{(\ba/2 - \bi/2)}%Breite Schenkel
\def\ly{(\ha/2 - \hi/2)}%Höhe Schenkel
\def\dyPrim{((\hi-\dy) / (\countPrim+1)/4)} %y-Verschiebung Wicklung Primärseite

\begin{circuitikz}[thick, every node/.style={transform shape, scale=1}, decoration={markings, mark=at position 0.5 with {\arrow{latex}}}]%global scale/.style={scale=1.0}, rotate=-5, xslant=-0.1
	\begin{scope}[even odd rule]
		\filldraw[rounded corners=2pt, fill=gray, rotate=-0, opacity=1.0] (\dx,\dy) rectangle ++(\ba,\ha) ({\lx+\dx},{\ly+\dy}) rectangle ++(\bi, \hi);%hinten
		\fill [rounded corners=2pt, fill=gray] (\ba, 0) --++ (0, \dy+\dr+\dr) --++(\dx, 0) --cycle;%rechte Seite
		\fill [rounded corners=2pt, fill=gray] (0, \ha) --++ (\dx+\dr+\dr, 0) --++(0, \dy)--cycle;%obere Seite
		\filldraw[rounded corners=2pt, fill=gray!50, rotate=-0] (0,0) rectangle ++(\ba, \ha) ({\lx},{\ly}) rectangle ++(\bi, \hi);%vorne
		\draw (\ba-\dr,\dr) --++(\dx, \dy);%Eckverbindungen
		\draw (\ba-\dr,\ha-\dr) --++(\dx, \dy);%Eckverbindungen
		\draw (\dr,\ha-\dr) --++(\dx, \dy);%Eckverbindungen
	\end{scope}



	%Richtungspfeile Fluss und Flächen
	\draw[magnetfeld, very thick, rounded corners=10pt, <-] ({\ba*0.4},{\ly/2}) -- ({\ba*0.55},{\ly/2});
	%Fläche unten
	\fill[fill=gray, fill opacity=1, blend mode=screen] ({\ba/2},0) -- ++(0,{\ly}) -- ++({\dx},{\dy}) -- ++(0,{-\ly});
	\draw ({\ba/2},0) node[below] {$A_4$};
	\draw[magnetfeld, very thick, rounded corners=10pt] ({\ba*0.55},{\ly/2}) -- ({\ba*0.7},{\ly/2}) node[right] {$\ell_4$};
	\draw[magnetfeld, very thick, rounded corners=10pt] ({\ba*0.8},{\ly/2}) -- ({\ba-\lx/2},{\ly/2}) -- ({\ba-\lx/2},{\ha*0.55});
	%Fläche rechts
	\fill[fill=gray, fill opacity=1, blend mode=screen] ({\ba-\lx+\dx},{\ha/(1.1+\dy)}) -- ++({\lx},0) -- ++({-\dx},{-\dy}) -- ++({-\lx},0);
	\draw ({\ba+\dx},{\ha/(1.1+\dy)}) node[right] {$A_3$};
	\draw[magnetfeld, very thick, rounded corners=10pt] ({\ba-\lx/2},{\ha*0.55}) -- ({\ba-\lx/2},{\ha*0.6}) node[above] {$\ell_3$};
	\draw[magnetfeld, very thick, rounded corners=10pt] ({\ba*0.55},{\ha-\ly/2}) -- ({\ba*0.45},{\ha-\ly/2}) node[left] {$\ell_2$};
	%Fläche oben
	\fill[fill=gray, fill opacity=1, blend mode=screen] ({\ba/2},{\ha-\ly}) -- ++(0,{\ly}) -- ++({\dx},{\dy}) -- ++(0,{-\ly});
	\draw ({\ba/2},{\ha-\ly}) node[below] {$A_2$};
	\draw[magnetfeld, very thick, rounded corners=10pt] ({\ba-\lx/2},{\ha*0.75}) -- ({\ba-\lx/2},{\ha-\ly/2}) -- ({\ba*0.55},{\ha-\ly/2});
	\draw[magnetfeld, very thick, rounded corners=10pt] ({\ba*0.35},{\ha-\ly/2}) -- ({\lx/2},{\ha-\ly/2}) -- ({\lx/2},{\ha*0.75}) node[below] {$\ell_1$};
	\draw[magnetfeld, very thick, rounded corners=10pt] ({\lx/2},{\ha*0.32}) -- ({\lx/2},{\ly/2}) -- ({\ba*0.3},{\ly/2}) node[right] {$\varPhi$};
	%Fläche links
	\fill[fill=gray, fill opacity=1, blend mode=screen] (\dx,{\ha/2.5}) -- ++({\lx},0) -- ++({-\dx},{-\dy}) -- ++({-\lx},0);
	\draw ({\dx+\lx},{\ha/2.5+0.2}) node[right] {$A_1$};
	\draw[magnetfeld, very thick, rounded corners=10pt] ({\lx/2},{\ha*0.6}) -- ({\lx/2},{\ha*0.32});

	%Primärwicklung
	%Zuleitung oben
	\draw[rounded corners=2pt, current, thick, postaction={decorate}] (-1,{\ly+\dr+\hi-\dy}) coordinate (U1oben)%Startpunkt oben
	-- ++(1,0) node[current, above, pos=0.4] {$\underline{I}$}; %Zuleitung
	%oberste Wicklung Primärwicklung
	\draw[rounded corners=2pt, current, thick]
	(0,{\ly+\dr+\hi-\dy}) %Startpunkt oben
	-- ++({\lx}, {-\dyPrim})% vordere Linie
	-- ++({\dx + 0.07},{-\dyPrim + \dy})%Linie nach hinten
	-- ++(-0.06,0.06); %Kreis rechts nach hinten
	%restliche Primärwicklungen
	\foreach \n in {2,...,\countPrim}
		{
			\draw[rounded corners=2pt, current, thick]
			(0,{\ly+\n*(\dyPrim * 4) + 0.08}) %Startpunkt oben
			-- ++(-0.07, -0.08)%Kreis links
			-- ++({\lx + 0.07}, {-\dyPrim})% vordere Linie
			-- ++({\dx + 0.07},{-\dyPrim + \dy})%Linie nach hinten
			-- ++(-0.06,0.06); %Kreis rechts nach hinten
		}
	%untere Zuleitung Primärwicklung
	\draw [current, thick, postaction={decorate}] (0, {\ly+((\hi-\dy) / (\countPrim+1))}) -- +(-1,0) coordinate (U1unten);
	%Primärspannung
	\draw (U1oben) to[voltage, open, v^=$\underline{U}$, o-o] (U1unten);
	\draw [-{Triangle[scale=1.2]}, color=voltage] (-1,2.4) -- (-1,1.44);
\end{circuitikz}
\begin{circuitikz}
	\draw [opacity=0](-1,0) -- ++(1,0); % Abstandhalter nach links
	\draw (0,0) coordinate (Uo) to[open, v=$\varTheta$] ++(0,-1) coordinate (Uu); % Quelle
	%\draw [-{Triangle[scale=1.2]}, color=voltage] (Uo) -- (Uu);		
	\draw (Uo) to[short, o-] ++(1,0) to[short] ++(0,0.5) coordinate (R2l);
	\draw (R2l) to[short] ++(0.5,0) to[R=$R_{m2}$] ++(2,0) coordinate (R2r); % R2
	\draw (R2r) to[color=magnetfeld, short, i=$\varPhi$] ++(0.5,0) coordinate (R3o);
	\draw (R3o) to[R=$R_{m3}$] ++(0,-3) coordinate (R3u); % R3
	\draw (R3u) to[R=$R_{m4}$] ++(-3,0) coordinate (R4l); % R4
	\draw (R4l) to[R=$R_{m1}$] ++(0,1.5) coordinate (R1o); % R1
	\draw (R1o) to[short, -o] (Uu);
\end{circuitikz}
	}{{\bf Magnetischer Kreis mit einem Eisenring.} Die vier magnetischen Teilwiderstände sind abhängig von der Länge, vom Material und dem Querschnitt des durchflossenen Eisenrings. \label{AbbMagnKreis}}

	\s{In einfachen Anordnungen (wie beispielsweise in Abbildung \ref{AbbMagnKreis} zu sehen) kann unter Vernachlässigung der Ecken der Widerstand durch die Gleichung \ref{GlmagnWiderstand} ausgedrückt werden. $\ell_{\mathrm{m}}$ ist wie bei der magnetischen Feldstärke die mittlere Feldlinienlänge des Widerstandes innerhalb des magnetischen Kreises. Zur Ermittlung der mittleren Feldlininenlänge ${\ell_{\mathrm{m}}}$ werden die Längen aller Seiten addiert.
	$A$ ist die Querschnittsfläche, die vom magnetischen Fluss durchflossen wird. $\mu$ ist die Permeabilität\index{Permeabilität} des Materials.
	Der magnetische Widerstand $R_{\mathrm{m}}$ berechnet sich nun aus der aufsummierten Länge der mittleren Feldlinienlänge ${\ell_{\mathrm{m}}}$ geteilt durch das Produkt der durchflossenen Querschnittsfläche $A$ und der Permeabilität $\mu_{\mathrm{r}} \cdot \mu_0$.
	
	\begin{eq}
		R_{\mathrm{m}} =\frac{\ell_{\mathrm{m}}}{\mu_{\mathrm{r}}\cdot\mu_0\cdot A} \qquad\left[\frac{\mathrm{A}}{\mathrm{V}\cdot\mathrm{s}}\right]\label{GlmagnWiderstand} 
	\end{eq}
	
	\begin{Merksatz}{Magnetischer Widerstand}
		Für die Berechnung des magnetischen Widerstands $R_{\mathrm{m}}$ wird die mittlere Feldlinienlängen ${\ell_{\mathrm{m}}}$ durch das Produkt aus der Permeabilität des Materials ${\mu_{\mathrm{r}}\cdot\mu_0}$ und der Querschnittsfläche $A$ geteilt. Bei unterschiedlicher Materialbeschaffenheit oder unterschiedlicher Querschnittsflächen innerhalb des magnetisierten Körpers werden zunächst die Teilwiderstände errechnet und anschließenden zum Gesamtwiderstand aufsummiert.
	\end{Merksatz}
	}
	\pause	
	\b{\begin{minipage}{0.5\textwidth}}
	\b{
		\begin{eq}
			R_{\mathrm{m}} =\frac{\ell_{\mathrm{m}}}{\mu_{\mathrm{r}}\cdot\mu_0\cdot A} \qquad\left[\frac{\mathrm{A}}{\mathrm{V} \cdot \mathrm{s}}\right]
		\end{eq}}
	\b{\end{minipage}}%
\end{frame}

\subsection{Der magnetische Kreis}
\begin{frame} \ftx{Magnetischer Kreis}\index{Magnetischer Kreis}		
	\b{
		Analog zum elektrischen Ohmschen Gesetz gilt für den magnetischen Kreis:
		\begin{eqa}
			\varTheta &= R_{\mathrm{m}}\cdot \varPhi
		\end{eqa}
	}
	\s{Werden nun alle bekannten magnetischen Größen in ihrer Zusammenwirkung betrachtet, ergibt sich die magnetische Analogie zum Ohmschen Gesetz. Sie beschreibt, dass die magnetische Spannung $\varTheta$  dem Produkt aus dem magnetischen Widerstand $R_{\mathrm{m}}$ und dem magnetischen Fluss $\varPhi$ entspricht.	
	\begin{eqa}
		\varTheta &= R_{\mathrm{m}}\cdot \varPhi
	\end{eqa}
	\begin{Merksatz}{Zusammenhänge der magnetischen Feldgrößen}


		Die magnetische Feldstärke $H$ im Spezialfall einer Ringkernspule errechnet sich aus dem Produkt der Windungsanzahl $N$ und dem Stromfluss $I$, geteilt durch die mittlere Feldlinienlänge $\ell_{\mathrm{m}}$. \par
		\begin{equation*}
			H=\frac{N\cdot I}{\ell_{\mathrm{m}}}
		\end{equation*}		
		Die magnetische Flussdichte $B$ wird durch  Multiplikation der magnetischen Feldstärke $H$ mit der Permeabilität $\mu_{\mathrm{r}} \cdot \mu_0$ bestimmt. \par
		\begin{equation*}	
			B = \mu_{\mathrm{r}} \cdot \mu_0 \cdot H 
		\end{equation*}	

		Der magnetische Fluss $\varPhi$ ergibt sich aus dem Integral der magnetischen Flussdichte $B$ über einer Fläche $A$.		\begin{equation*} 
			\varPhi = \iint\limits_A \vec{B} \cdot \mathrm{d} \vec{A}  
		\end{equation*}	


	\end{Merksatz}
	}
\end{frame}

