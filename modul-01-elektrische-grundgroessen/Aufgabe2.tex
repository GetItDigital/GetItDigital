\subsection{Elektrisches Homogenfeld 2\label{ElHomFeld2}}
\Aufgabe{
	Zwischen den Platten des nebenstehenden Kondensators befinden sich zwei Isolierstoffe mit den Dielektrizitätszahlen $\varepsilon_{\mathrm{r1}}$ und $\varepsilon_{\mathrm{r2}}$. Der Kondensator ist über den
    Schalter $S$ mit einer Batterie verbunden.\\
    \newline
    Gegeben sind die Größen: Plattenteilflächen $A_1 = A_2 = 100 \ cm^2; U_{\mathrm{B}}=100 \ V;d = 1 \ cm$.
    \begin{enumerate}[label=\alph*)]
            \item  Der Schalter ist geschlossen. Es gelte $\varepsilon_{\mathrm{r1}}=\varepsilon_{\mathrm{r2}}=1$. Wie groß sind die Kondensatorladung $Q$, die elektrische Flussdichte $D$, die elektrische Feldstärke $E$
            sowie die Kapazität $C_{\mathrm{AB}}$?
            \item Der Schalter ist geschlossen. Es gelte $\varepsilon_{\mathrm{r1}}=1, \varepsilon_{\mathrm{r2}}=3$. Wie hat sich die Gesamtkapazität allgemein im Vergleich zu Punkt a) verändert? Zu bestimmen
            sind außerdem die Werte $Q_1$ und $Q_2$, $D_1$ und $D_2$ sowie $E_1$ und $E_2$. Welche Spannung $U_{AB}$ stellt sich am Kondensator ein?
            \item  In die Anordnung gemäß Punkt a) wird \underline{nach} dem Öffnen des Schalters $S$ ein anderes Dielektrikum eingeführt, so dass gilt: $\varepsilon_{\mathrm{r1}}=1, \varepsilon_{\mathrm{r2}}=3$.
            Welche Spannung $U_{\mathrm{AB}}$ stellt sich am Kondensator ein?
    \end{enumerate}
}

\Loesung{
	Hier entsteht eine Musterlösung...
}