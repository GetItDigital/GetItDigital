\subsection{Elektrisches Homogenfeld 3\label{ElHomFeld 3}}
\Aufgabe{
    Mit einem Dielektrikum, in dem maximal eine Feldstärke $E = 30 \ \frac{kV}{cm}$ zulässig ist, soll ein Kondensator für den Spannungsbereich von $100 \ kV$ realisiert werden.\\
    \newline
    Das Dielektrikum wird zunächst in einem (idealen) \underline{Plattenkondensator} eingesetzt.\\
    \begin{enumerate}[label=\alph*1)]
        \item Skizzieren Sie die Anordnung.
        \item Skizzieren Sie die Feldlinien in der Anordnung.
        \item Wo ist der Ort der maximalen Feldstärke?
        \item Skizzieren Sie den Verlauf des Betrags der elektrischen Feldstärke von einer Kondensatorplatte zur anderen $\left| E(x) \right|$, parallel zu den Kondensatorplatten
        $\left| E_{\mathrm{Diel}}(z) \right|$ und in den Kondensatorplatten $\left| E_{\mathrm{Pl}}(z) \right|$.
        \item Zeichnen Sie die 50\%-Äquipotentiallinie ein.
        \item Berechnen Sie den minimalen Plattenabstand $d_{\mathrm{Pl,min}}$ für die o.a. Randbedingungen.            
    \end{enumerate}
    Nun soll das \underline{Dielektrikum zwischen koaxialen Zylindern} eingesetzt werden.\\
    \newline
    $\left(E_{\mathrm{r}}(r)=\frac{U}{r\cdot\ln\left(\frac{a}{b}\right)};\ r_{\mathrm{a}}: Außenradius; r_{\mathrm{i}}: Radius \ des \ Innenzylinders\right)$
    \begin{enumerate}[label=\alph*2)]
        \item Skizzieren Sie die Anordnung.
        \item Skizzieren Sie die Feldlinien.
        \item Wo ist der Ort der maximalen Feldstärke?
        \item Skizzieren Sie den Verlauf der elektrischen Feldstärke $E_{\mathrm{r}}(r)$ von $r=r_{\mathrm{i}}$ bis $r=r_{\mathrm{a}}$.
        \item Zeichnen Sie die 50\%-Äquipotentiallinie ein.
        \item Berechnen Sie den minimalen Außenradius $r_{\mathrm{a}}$ für $r_{\mathrm{i}} = 1;\ 2;\ 3;\ 4;\ 5;\ 10 \ cm$ und
        die zugehörige Dicke des Dielektrikums $d_{\mathrm{i}}$.                        
    \end{enumerate}
}

\Loesung{
	Hier entsteht eine Musterlösung...
}