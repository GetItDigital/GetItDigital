\subsection{Ladungsträgergeschwindigkeit\label{LadungGeschw}}
\Aufgabe{
    Durch eine Kupferleitung mit dem Querschnitt $A = 1 \ mm^2$ und mit der Länge $L = 10 \ m$ fließt ein Strom $I = 8 \ A$. Ein $mm^3$ Kupfer enthält $8,5\cdot10^{19}$ Atome.
    Es darf davon ausgegangen werden, dass jeweils 1 Elektron pro Atom am Ladungstransport beteiligt ist $(\vartheta = 20 \ ^\circ C)$.
    \newline
    \begin{enumerate}[label=\alph*)]
        \item Bestimmen Sie die Driftgeschwindigkeit der Elektronen in der Kupferleitung.
        \item Welche elektrische Feldstärke $E$ besteht in der Kupferleitung?
        \item Wie hoch ist der Spannungsabfall $U$ in dieser Kupferleitung?
        \item Welchen Widerstand hat die Kupferleitung bei den angegebenen Randbedingungen? Welchen Widerstand nimmt der Draht bei einer Erwärmung auf $\vartheta_{\mathrm{w}} = 180 \ ^\circ C$ an?
        Wie groß ist die prozentuale Widerstandserhöhung?
    \end{enumerate}
}

\Loesung{
	Hier entsteht eine Musterlösung...
}