\subsection{Elektromagnetismus\label{SectionElektromagentismus}}

\s{Ein magnetisches Feld kann, außer durch einen Permanentmagneten, auch durch einen elektrischen Strom erzeugt werden. Dabei ergibt sich ein kreisförmiges magnetisches Feld um den stromdurchflossenen Leiter.
	Die Richtung des Feldes kann mit der \glqq Rechten-Hand-Regel\grqq \index{Rechte Hand Regel} ermittelt werden: Bei einer zur Faust geformten Hand mit gestrecktem Daumen zeigt der Daumen in Richtung des (technischen) elektrischen Stromflusses (der \glqq technische Stromfluss\grqq \,verläuft vom Plus- zum Minuspol) und die Finger in Richtung des magnetischen Feldlinienverlaufs.
	\
	\begin{Merksatz}{Rechte-Hand-Regel}
		Zeigt der Daumen der rechten Hand in Richtung des (technischen) elektrischen Stromflusses, so zeigen die restlichen gebeugten Finger die Verlaufsrichtung der magnetischen Feldlinien an.
	\end{Merksatz}
}

\begin{frame} \ftx{Elektromagnetismus}
	\s{
		\fo{width=\textwidth}{width=\textwidth}{Magnetisches_Feld_Leiter_und_Leiterschleife}{
			\put(29, 5){\color{current}$I$}
			\put(67.5, 9){\color{current}$I$}
			\put(83.5, 2.5){\color{current}$I$}
			\put(55, 16.8){\color{magnetfeld}N}
			\put(91, 37.4){\color{magnetfeld}S}
		}{{\bf Magnetisches Feld um einen stromdurchflossenen Leiter.} Links: Die Rich\-tung des Feldes kann mit der \glqq Rechten-Hand-Regel\grqq\ veranschaulicht werden. Rechts: Die Leiterschleife stellt die kleinste Einheit einer Spule dar. Die hier dargestellte mittlere Feldlinie schließt sich in der Unendlichkeit.}
	}
	\b{
		\begin{minipage}{0.45\textwidth}
			\fo{width=\textwidth}{width=\textwidth}{Magnetisches_Feld_Leiter}{
				\put(60.5, 9){\color{current}$I$}
			}{}
		\end{minipage}
		\hfill\pause
		\begin{minipage}{0.45\textwidth}
			\fo{width=\textwidth}{width=\textwidth}{Magnetisches_Feld_Leiterschleife}{
				\put(31.5, 18){\color{current}$I$}
				\put(65, 4){\color{current}$I$}
				\put(5, 37){\color{magnetfeld}N}
				\put(80, 79){\color{magnetfeld}S}
			}{}
		\end{minipage}
	}
\end{frame}
\nomenclature[EA]{A}{Ampere - elektrischer Strom}
\s{\subsection{Durchflutung\label{durchflutung}}}
\begin{frame} \ftx{Magnetische Durchflutung}
	\index{Magnetische Durchflutung}\index{Durchflutung}
	\s{Die durch elektrische Ströme hervorgerufenen Magnetfelder werden mit der Kenngröße der magnetischen Durchflutung $\varTheta$ (Theta) gemessen.
		Analog zur elektrischen Spannung wird die Durchflutung auch als magnetische Spannung bezeichnet. 
		Da die Durchflutung aus dem Strom resultiert, hat sie wie der Strom die Einheit Ampere.}
	
	\b{
		\begin{minipage}{0.54\textwidth}
			\fo{}{width=\textwidth}{Magnetisches_Feld_Spule}{
				\put(5, 65){\color{current}$I$}
				\put(26, 73){\color{current}$I$}
				\put(2, 8){\parbox{3cm}{Elektrische\\Durchflutung\\= $N \cdot I$}}
			}{}
		\end{minipage}
		\begin{minipage}{0.45\textwidth}
			Durchflutung oder magnetische Spannung $\varTheta$ (Theta) = Gesamtstrom einer durchfluteten Fläche
			\begin{eqa}
				\onslide<2->\varTheta &= \sum_{\mathrm{n}} I_{\mathrm{n}} = N\cdot I\qquad\text{(vereinfacht)}\\
				\onslide<3->\varTheta &= \iint\limits_A \vec{J}\cdot\d\vec{A} \qquad\text{(allgemein)}\\
				\onslide<4->\varTheta &= \oint\limits_s \vec{H}\cdot \d\vec{s} \qquad{\left[\mathrm{A}\right]}
				\onslide<1->
			\end{eqa}
			\onslide<4->$H$ ist die magnetische Feldstärke\\
			$s$ ist der Rand der Fläche $A$
			\onslide<1->
		\end{minipage}
	}
	
	\s{
		Der Name ist auf das Durchflutungsgesetz zurückzuführen. Dieses besagt, dass die magnetische Durchflutung $\varTheta$ gleich dem Gesamtstrom $I$ einer von ihm durchfluteten Fläche ist (siehe Abbildung \ref{Magnet5}).
		Die Gleichnung \ref{Durchflutung} gibt das Durchflutungsgesetz in der allgemeinen Form wieder. Die rechte Seite bezeichnet das Flächenintegral der Stromdichte $\vec{J}$. Dieses Integral drückt die Summe des Stromes aus, der durch die Fläche $A$ fließt. 
		Die linke Seite der Gleichung stellt das geschlossene Linienintegral über die magnetische Feldstärke $\vec{H}$ dar. Diese geschlossene Linie $\d\vec{s}$ entspricht dabei dem Rand der Fläche $A$.
		
		\begin{equation}
			\varTheta = \oint\limits_s \vec{H}\cdot \d\vec{s} = \iint\limits_A \vec{J}\,\d\vec{A}\label{Durchflutung}
		\end{equation}
		
		Da in den allermeisten Fällen der Strom durch einen Leiter transportiert wird und folglich die Richtung des Stromes als auch die Stromstärke eindeutig bekannt sind, genügt es in diesem Fall, das Integral (wie in der Gleichung \ref{Durchflutungsgesetz} dargestellt) durch die Anzahl der stromführenden Leiter $N$ multipliziert mit der Stromstärke $I$ zu ersetzen.
		
		\begin{eq}
			\varTheta = \oint\limits_s \vec{H}\cdot \d\vec{s} = N\cdot I\label{Durchflutungsgesetz} \qquad [\mathrm{A}]
		\end{eq}
		
		\begin{Merksatz}{Magnetische Durchflutung}
			Die magnetische Durchflutung $\varTheta$ entspricht dem Gesamtstrom einer von ihm durchfluteten Fläche.
		\end{Merksatz}
		
		\fo{width=0.5\textwidth}{}{Magnetisches_Feld_Spule}{
			\put(5, 66){\color{current}$I$}
			\put(26, 73){\color{current}$I$}
			\put(20.5, 39.2){\line(1,1){15}}\put(-5, 37){\parbox{2cm}{Durchflutete Fläche}}
			\put(20.9, 16.6){\line(1,1){10.8}}\put(-5, 17){\parbox{2cm}{Daraus\\resultierende Feldlinie}}
			\put(98, 5){\parbox{3cm}{Elektrische\\Durchflutung\\= $N \cdot I$}}
		}{{\bf Durchflutung einer Spule.} Die Durchflutung entspricht dem Gesamtstrom durch eine durchflutete Fläche. Die elektrische Durchflutung einer Spule ist daher abhängig von der Wicklungszahl und der Stromstärke. Für die Darstellung in einer zweidimensionalen Grafik wird das Symbol $\otimes$ für den Stromfluss \glqq in die Bildfläche hinein\grqq, $\odot$ für \glqq aus der Bildfläche heraus\grqq\ verwendet. \label{Magnet5}}
		\index{Magnetischer Fluss}
	}
\end{frame}


\begin{frame} \ftx{Magnetische Feldstärke}
	\index{Magnetische Feldstärke}
	
	\s{
		Ist die magnetische Feldstärke über dem Integrationsweg $\d\vec{s}$ konstant, kann der Vektor $\vec{H}$ vor das Integral gezogen werden. Diese Bedingung ist in der Regel erfüllt, wenn sich die Feldlinie über dem ganzen Integrationsweg im gleichen Material befindet. 
		Ein Beispiel wäre ein kreisförmiges Feld um einen Leiter im Kreismittelpunkt oder eine Ringkernspule, wie in Abbildung \ref{Magnet6} gezeigt. 
		In dem Fall wird das Integral $\oint \vec{H}\cdot\,\d\vec{s}$ zur Länge des Integrationsweges (Gleichung \ref{GlmagnFeldstaerke}). Diese Länge wird mittlere Feldlinienlänge genannt und mit $\ell_{\mathrm{m}}$ bezeichnet. Sie steht für den Mittelwert der Summe aller Feldlinien, die sich innerhalb des kreisförmigen Feldes befinden. Die magnetische Feldstärke kann in solchen Anordnungen einfach mit der Gleichung \ref{GlmagnFeldstaerke1} ermittelt werden.
		
		\begin{eqa}
			\varTheta &= \oint\limits_s \vec{H}\cdot \d\vec{s} = |\vec{H}| \cdot \ell_{\mathrm{m}}\label{GlmagnFeldstaerke}\\
			|\vec{H}| &=\frac{\varTheta}{\ell_{\mathrm{m}}}\qquad\left[\frac{\mathrm{A}}{\mathrm{m}}\right]\label{GlmagnFeldstaerke1}
		\end{eqa}
		
		\begin{Merksatz}{Berechnungshilfe magnetische Feldstärke}
			Ist die magnetische Feldstärke über den ganzen Weg konstant, kann sie durch den Quotienten von Durchflutung $\varTheta$ und Weglänge $\ell_{\mathrm{m}}$ berechnet werden. Ihre Einheit ist Ampere pro Meter.
		\end{Merksatz}
		
		\fo{width=0.45\textwidth}{width=0.5\textwidth}{Magnetfeldline_Ringspule}{
			\put(32, 6){\color{current}$I$}
			\put(65, 10){\color{current}$I$}
			\put(85, 17.5){\line(-1,1){15}}\put(85.3, 17){\parbox{3cm}{Mittlere\\Feldlinienlänge $\ell_{\mathrm{m}}$}}
		}{{\bf Ringkernspule mit mittlerer Feldlinienlänge.} Die mittlere Feldlinienläge repräsentiert den Mittelwert aller Feldlinien innerhalb der Spule, womit die Feldstärke einfach rechnerisch ermittelt werden kann. \label{Magnet6}}
	}
	\b{
		\begin{minipage}{0.54\textwidth}
			\vspace{3.5em}
			\fo{}{width=0.85\textwidth}{Magnetfeldline_Ringspule}{
				\put(30.5, 6){\color{current}$I$}
				\put(66, 10){\color{current}$I$}
				\put(85, 17.5){\line(-1,1){15}}\put(77, 7){\parbox{3cm}{Mittlere\\Feldlinienlänge $\ell_{\mathrm{m}}$}}
			}{}
		\end{minipage}
		\begin{minipage}{0.4 \textwidth}
			\vspace{-3.5em}
			Wenn $\vec{H}$ über dem Integrationsweg $\d\vec{s}$ konstant ist:
			\begin{flushleft}
				\begin{eqa}
					\oint\limits_s \vec{H}\cdot \d\vec{s} = |\vec{H}| \cdot \ell_{\mathrm{m}}\\
					|\vec{H}| =\frac{\varTheta}{\ell_{\mathrm{m}}}\qquad \left[\frac{\mathrm{A}}{\mathrm{m}}\right]
				\end{eqa}
			\end{flushleft}
			
		\end{minipage}
	}
	
\end{frame}

\begin{frame} \ftx{Beispiel: Magnetische Feldstärke}
	\begin{bsp}{Magnetische Feldstärke über die mittlere Feldlinienlänge}{}
		\s{Eine Ringspule (Abbildung \ref{Magnet6}) mit $1000$ Windungen und einer mittleren Feldlinienlänge von $50\,\mathrm{cm}$ wird von einer Stromstärke von $100\,\mathrm{mA}$ durchflossen. Wie groß ist die magnetische Feldstärke?}%
		\b{Eine Ringspule mit $1000$ Windungen und einer mittleren Feldlinienlänge von $50\,\mathrm{cm}$ wird von einer Stromstärke von $100\,\mathrm{mA}$ durchflossen. Wie groß ist die magnetische Feldstärke?}%
		\begin{eqa}
			\onslide<2->\varTheta &= H \cdot \ell_{\mathrm{m}} = N\cdot I\nonumber\\
			\onslide<3->H&=\frac{N\cdot I}{\ell_{\mathrm{m}}}\onslide<4->=\frac{1000\cdot 0,1\,\mathrm{A}}{0,5\,\mathrm{m}} = 200\,\tfrac{\mathrm{A}}{\mathrm{m}}\nonumber
			\onslide<1->
		\end{eqa}
	\end{bsp}\onslide<5->
	
	\begin{bsp}{Magnetische Feldstärke über den Kreisumfang}{}
		Ein gerader Leiter wird mit einem Strom von $I=50\,\mathrm{A}$ durchflossen. Wie groß ist die magnetische Feldstärke in einem Abstand von $r=20\,\mathrm{cm}$?
		\begin{equation*}
			\onslide<6->H=\frac{N\cdot I}{\ell_{\mathrm{m}}}\onslide<7->=\frac{1\cdot50\,\mathrm{A}}{2\pi\cdot 0,2\,\mathrm{m}} = 39,79\,\tfrac{\mathrm{A}}{\mathrm{m}}
		\end{equation*}
		\s{Die mittlere Feldlinienlänge $\ell_{\mathrm{m}}$ kann durch den Kreisumfang ermittelt werden und wird deshalb mit der Formel zur Berechnung des Kreisumfangs $2\pi\cdot r$ ersetzt.}
	\end{bsp}
\end{frame}

\s{\subsection{Magnetischer Fluss und Flussdichte\label{Flussdichte}}}
\begin{frame} \ftx{Magnetischer Fluss und Flussdichte}
	\index{Magnetische Flussdichte}\index{Flussdichte}\index{Magnetischer Fluss}
	\s{
		Der magnetische Fluss ist analog zum elektrischen Kreis mit dem Strom vergleichbar. Entgegen der Terminologie findet jedoch kein Fluss von magnetischen Teilchen statt, sondern er ist sinnbildlich als \glqq die Menge an Magnetfeld\grqq\,zu verstehen und wirkt in Folge der magnetischen Spannung. Das Formelzeichen des magnetischen Flusses ist $\varPhi$, die Einheit ist Weber (Wb). Ein Weber (Wb) ist gleichbedeutend mit einer Volt-Sekunde (Vs).
		
		Die Kraftwirkung eines Magneten ist abgesehen vom magnetischen Fluss auch von der durchfluteten Fläche abhängig. Je dichter die Feldlinien konzentriert sind, desto größer ist die magnetische Wirkung. Das wird durch die magnetische Flussdichte $B$ beschrieben, die im einfachsten Fall (nicht gekrümmte Fläche, homogene Flussdichte) durch den Quotienten aus dem magnetischen Fluss $\varPhi$ und der Fläche $A$ definiert ist. Die Einheit der Flussdichte ist Tesla (T). Die Richung der Flussdichte $\vec{B}$ ist senkrecht zur Fläche, was durch den Normalenvektor zur Fläche $\vec{A}$ ausgedrückt wird.
		
		\begin{eq}
			\vec{B} = \frac{\varPhi}{\vec{A}}\qquad [\mathrm{T}]
		\end{eq}
		\nomenclature[ET]{T}{Tesla - magnetische Flussdichte \nomunit{$\mathrm{T}=\frac{\mathrm{V}\cdot\mathrm{s}}{\mathrm{m}^2}=\frac{\mathrm{kg}}{\mathrm{s}^2\cdot\mathrm{A}}$}}
		
		Im allgemeinen Fall (ohne die oben genannten Einschränkungen) gilt:
		
		\begin{eq}
			\varPhi = \iint\limits_A \vec{B}\cdot\d \vec{A} \label{magnFlussFormel}
		\end{eq}
		\begin{Merksatz}{Magnetische Flussdichte}
			Die magnetische Flussdichte $\vec{B}$ beschreibt die Konzentration des magnetischen Flusses $\varPhi$ senkrecht auf einer Fläche $A$.
		\end{Merksatz}
		
		Sowohl die magnetische Feldstärke als auch die magnetische Flussdichte sind vektorielle Größen. Sie lassen sich daher grafisch durch Feldlinien zeichnen. Der magnetische Fluss $\varPhi$ ist dagegen eine skalare Größe.
		
		Die magnetische Flussdichte $\vec{B}$ und die magnetische Feldstärke $\vec{H}$ sind über die Permeabilität\index{Permeabilität} $\mu$ verbunden. Die Permeabilität besteht aus dem Produkt einer materialunabhängigen Kenngröße, der magnetischen Feldkonstanten $\mu_0=1,256\,637\,062\cdot10^{-6}\,\frac{\mathrm{Vs}}{\mathrm{Am}}$, und einer materialspezifischen Permeabilität $\mu_{\mathrm{r}}$. Die magnetische Feldkonstante beschreibt die Permeabilität im Vakuum und war bis zur Neuordnung der SI-Einheiten im Jahr 2019 mit dem Wert $\mu_0 = 4\pi\cdot10^{-7}\,\frac{\mathrm{Vs}}{\mathrm{Am}}$ genau definiert. Jetzt ist sie mit einer Messunsicherheit behaftet.

	}
	\s{
		\begin{table}[H]
			\centering
			\caption{{\bf Materialabhängige Permeabilität.} Exemplarische Übersicht zur Permeabilitätszahl $\mu_{\mathrm{r}}$ unterschiedlicher Materialien:}
			\begin{tabular}{lc}
				\bf{Material}       & \bf{Permeabilitätszahl $\mu_{\mathrm{r}}$} \\
				\midrule
				Wasser    & $1 - 9,1 \cdot 10^{-6}$ \\
				Kupfer    & $1 - 6,4 \cdot 10^{-6}$ \\
				Luft      & $1 + 4 \cdot 10^{-7}$ \\
				Aluminium & $1 - 2,2 \cdot 10^{-5}$ \\
				Eisen     & $300$ bis $140000$ \\
			\end{tabular}%
		\end{table}
		\begin{eq}
			\vec{B} = \mu_0\cdot\mu_{\mathrm{r}}\cdot \vec{H}\label{GlFlussdichte}
		\end{eq}
	}
	\b{
		\begin{columns}
			\begin{column}{0.7\textwidth}
				\begin{minipage}{0.8\textwidth}
					% Nicht linearer Zusammenhang zwischen B und H -> Neukurve und Sättigung
					\onslide<1->{
						\fu{
							\begin{tikzpicture}[domain=-6:6, scale=0.73]
								\draw[->] (-6.1, 0) -- (6.2, 0) node[right] {$H$};
								\draw[->] (0, -5.1) -- (0, 5.2) node[above] {$B$};
								\draw (-0.2, 0) node[below] {0};
								
								\onslide<3->{
									\draw[color=blue, smooth, domain=0:6] plot[id=hysterese1] function{(((x-6.1)**3)/50)+4.485};
									\draw[->, color=blue, thick] (1, 1.82) -- (1.05, 1.91);
									\draw (6, 4.85) node[left] {Sättigung};
									\draw[color=blue] (2.2, -1.5) node[right] {Neukurve};
								}
								% Magnetisierung bleibt bei Reduktion erhalten (Punkt Br)
								\onslide<4->{
									\draw[color=red, smooth] plot[id=hysterese2] function{9/(1+0.4*exp(-1.1*x))-4.5};
									\draw[->, color=red, thick] (-0.55, 0.7) -- (-0.61, 0.55);
									\draw (-6, -4.85) node[right] {Sättigung};
									\draw (-0.2, 1.9) node[left] {$B_{\mathrm{r}}$} -- (0.2, 1.9);
								}
								% Gilt ebenso in andere Richtung -> Hystereseschleife
								\onslide<5->{
									\draw[color=red, smooth] plot[id=hysterese3] function{9/(1+2.5*exp(-1.1*x))-4.5};
									\draw[->, color=red, thick] (0.55, -0.7) -- (0.61, -0.55);
									\draw[color=red] (2.2, -2.3) node[right] {Hystereseschleife};
									\draw (-0.2, -1.9) -- (0.2, -1.9);
									
								}
								%Zum Entmagnetisieren umgekehrte Koeffizienzfeldstärke (Hc)
								\onslide<6->{
									\draw (-0.85, 0.2) -- (-0.85, -0.2);
									\draw (-1.1, 0.2);
									\draw (0.85, 0.2) -- (0.85, -0.2);
									\draw (1.1, -0.2) node[below] {$H_{\mathrm{c}}$};
								}
							\end{tikzpicture}
						}{}
					}
				\end{minipage}
			\end{column}
			
			\begin{column}{0.3\textwidth}
				\only<1-2>{Magnetische Flussdichte: \par  $\vec{B} \quad \left[\mathrm{T}\right]$\\
					Permeabilität: \par 
					$\mu\quad \left[1\right]$
					\begin{eq}
						\vec{B} = \mu_0\cdot\mu_{\mathrm{r}}\cdot \vec{H}
					\end{eq}
					
					\onslide<2->{
						Magnetischer Fluss: \par $\varPhi\quad \left[\mathrm{Wb}\right]$
						\begin{align*}
							\varPhi & = \iint\limits_A \vec{B}\cdot\d \vec{A} \\
							\vec{B} & = \frac{\varPhi}{\vec{A}}
						\end{align*}}
				}
				\only<3->{%
					
					Hartmagnetisch: \\
					$H_{\mathrm{c}} > 10\cdot 10^{3}\,\frac{\mathrm{A}}{\mathrm{m}}$\\
					z. B. Permanentmagnete\\
					\vspace{0.5cm}
					Weichmagnetisch:\\
					$H_{\mathrm{c}} < 500\,\frac{\mathrm{A}}{\mathrm{m}}$\\
					z. B. Aktoren\\
					\vspace{0.5cm}
				}
				\only<4->{
					Remanenzflußdichte $B_{\mathrm{r}}$\\
					\vspace{0.5cm}
				}
				\only<6->{
					Koerzitivfeldstärke $H_{\mathrm{c}}$
				}
				
			\end{column}
		\end{columns}
	}
	\s{
		In einem ferromagnetischen\index{Ferromagnetismus} Material verläuft der Zusammenhang zwischen der magnetischen Feld\-stärke $\vec{H}$ und der magnetischen Flussdichte $\vec{B}$ nicht linear. 
		Die Permeabilität geht mit steigender Magnetisierung in Sättigung, sodass die relative Permeabilität $\mu_{\mathrm{r}}$ von einem materialabhängigen Anfangswert gegen 1 läuft. 
		Wird das magnetische Feld wieder reduziert (oder auf Null gesetzt), bleibt die Magnetisierung in einem gewissen Maße erhalten (Punkt $B_{\mathrm{r}}$ in Abbildung \ref{Hysterese}). 
		Dieser Vorgang wird Remanenz\index{Remanenz} genannt. Die verbleibende magnetische Flussdichte bei einer magnetischen Feldstärke von Null ist die Remanenzflussdichte $B_{\mathrm{r}}$. 
		Um den Stoff wieder komplett zu untmagnetisieren\index{Entmagnetisieren}, wird eine umgekehrte magnetische Feldstärke, die Koerzitivfeldstärke\index{Koerzitivfeldstärke} $H_{\mathrm{c}}$, benötigt.\index{Magnetisierungskennlinie}
		\fu{
			\begin{tikzpicture}[domain=-6:6, scale=0.85]
	\draw[->] (-6.1, 0) -- (6.2, 0) node[right] {$H$};
	\draw[->] (0, -5.1) -- (0, 5.2) node[above] {$B$};
	\draw (-0.2, 0) node[below] {0};
	\draw[color=blue, smooth, domain=0:6] plot[id=hysterese1] function{(((x-6.1)**3)/50)+4.485};
	\draw[->, color=blue, thick] (1, 1.82) -- (1.05, 1.91);
	\draw[color=red, smooth] plot[id=hysterese2] function{9/(1+0.4*exp(-1.1*x))-4.5};
	\draw[->, color=red, thick] (-0.55, 0.7) -- (-0.61, 0.55);
	\draw[color=red, smooth] plot[id=hysterese3] function{9/(1+2.5*exp(-1.1*x))-4.5};
	\draw[->, color=red, thick] (0.55, -0.7) -- (0.61, -0.55);
	\draw[color=blue] (2.5, -1.5) node[right] {Neukurve};
	\draw[color=red] (2.5, -2.1) node[right] {Hystereseschleife};
	\draw (6, 4.85) node[left] {Sättigung};
	\draw (-6, -4.85) node[right] {Sättigung};
	\draw (-0.2, 1.9) node[left] {$B_{\mathrm{r}}$} -- (0.2, 1.9);
	\draw (-0.2, -1.9) -- (0.2, -1.9);
	\draw (-0.85, 0.2) -- (-0.85, -0.2);
	\draw (-1.1, 0.2);
	\draw (0.85, 0.2) -- (0.85, -0.2);
	\draw (1.1, -0.2) node[below] {$H_{\mathrm{c}}$};
\end{tikzpicture}
		}{{\bf Magnetisierungskurve eines ferromagnetischen Materials.} Die magnetische Feldstärke und die magnetische Flussdichte verhalten sich nicht linear zueinander. \label{Hysterese}}
		
		Anhand der Koerzitivfeldstärke werden ferromagnetische Materialien in hart- und weichmagnetische Materialien unterschieden. Hartmagnetische Werkstoffe (z. B. starke Dauermagnete aus Neodym-Eisen-Bor) verfügen dabei über einen Wert für $H_{\mathrm{c}}$ größer als $10\cdot 10^{3}\,\frac{\mathrm{A}}{\mathrm{m}}$, bei weichmagentischen Werkstoffen (z. B. Magnetkerne aus Mangan-Zink-Ferrit) liegt $H_{\mathrm{c}}$ bei kleiner als $500\,\frac{\mathrm{A}}{\mathrm{m}}$. Hartmagnetische Werkstoffe werden hauptsächlich für Permanentmagnete eingesetzt.\\
		
	}
\end{frame}

\begin{frame} \ftx{Beispiel: Magnetische Flussdichte}
	\begin{bsp}{Magnetische Flussdichte}{}
		Im Inneren einer dicht gewickelten Ringspule soll die magnetische Feldstärke $H=100\,\frac{\mathrm{A}}{\mathrm{m}}$ erzeugt werden. Die Spule hat einen mittleren Radius von $5\,\mathrm{cm}$.
		\begin{enumerate}
			\item Berechnen Sie die erforderliche Stromstärke $I$ wenn die Spule mit $N=200$ Wicklungen versehen ist.\pause
			      
			      \s{Aus Gleichung \ref{Durchflutungsgesetz} und \ref{GlmagnFeldstaerke}:}
			      \begin{align*}
				      \varTheta & = H \cdot \ell_{\mathrm{m}} = N\cdot I                                                                                                                  \\
				      I         & =\frac{H\cdot\ell_{\mathrm{m}}}{N}=\frac{100\,\frac{\mathrm{A}}{\mathrm{m}}\cdot 2\cdot \pi\cdot 5\cdot 10^{-2}\,\mathrm{m}}{200} = 157,08\,\mathrm{mA}
			      \end{align*}\pause
			\item Wie groß wird die Flussdichte $B$ im Falle einer Luftspule ($\mu_{\mathrm{r}}=1$) oder einer eisengefüllten Spule ($\mu_{\mathrm{r}}=2000$ im Arbeitspunkt)?\pause
			      
			      \s{Aus Gleichung \ref{GlFlussdichte}:}
			      \begin{align*}
				      B_\mathrm{Luft}  & = \mu_0\cdot \mu_{\mathrm{r}}\cdot H = 1,256\cdot10^{-6}\,\tfrac{\mathrm{Vs}}{\mathrm{Am}} \cdot 100\,\tfrac{\mathrm{A}}{\mathrm{m}} = 125,6\,\mu\mathrm{T} \\
				      B_\mathrm{Eisen} & = 2000\cdot B_\mathrm{Luft} = 251,2\,\mathrm{mT}
			      \end{align*}
		\end{enumerate}
	\end{bsp}
\end{frame}

\nomenclature[FT]{$t$}{Zeit \nomunit{s}}
\nomenclature[FL]{$\ell$}{Länge \nomunit{m}}
\nomenclature[FM]{$m$}{Masse \nomunit{kg}}
\nomenclature[FI]{$I$}{elektrischer Strom (zeitlich konstant) \nomunit{A}}
\nomenclature[FU]{$U$}{Spannung (zeitlich konstant)\nomunit{V}}
\nomenclature[FU]{$u$}{Spannung (zeitlich variabel)\nomunit{V}}
\nomenclature[F0]{$\varPhi$}{magnetischer Fluss \nomunit{Wb}}
\nomenclature[FB]{$B$}{magnetische Flussdichte \nomunit{T}}
\nomenclature[FN]{$N$}{abstrakte Anzahl (z. B. Anzahl der Wicklungen einer Spule) \nomunit{$1$}}
\nomenclature[FE]{$\vec{e}_{\mathrm{r}}$}{Einheitsvektor in die Richtung r. $|\vec{e}_\mathrm{r}|=1$}
\nomenclature[FM]{$M$}{Drehmoment \nomunit{$\mathrm{Nm}$}}
\nomenclature[FA]{$A$}{Fläche \nomunit{$\mathrm{m}^2$}}
\nomenclature[FD]{$d$}{Abstand im magnetischen Leitkörper \nomunit{m}}
\nomenclature[F0]{$\varTheta$}{magnetische Durchflutung \nomunit{A}}
\nomenclature[FJ]{$\vec{J}$}{Stromdichte \nomunit{$\frac{\mathrm{A}}{\mathrm{m}^2}$}}
\nomenclature[FL]{$\ell_{\mathrm{m}}$}{mittlere Feldlinienlänge einer Spule \nomunit{m}}
\nomenclature[FH]{$H$}{magnetische Feldstärke \nomunit{$\frac{\mathrm{A}}{\mathrm{m}}$}}
\nomenclature[FR]{$R_{\mathrm{m}}$}{magnetischer Widerstand \nomunit{$\frac{\mathrm{A}}{\mathrm{V}\cdot\mathrm{s}}$}}
\nomenclature[FV]{$v$}{Geschwindigkeit \nomunit{$\frac{\mathrm{m}}{\mathrm{s}}$}}
\nomenclature[FW]{$W$}{Arbeit \nomunit{$\mathrm{J}$}}
\nomenclature[C0]{$\mu_{\mathrm{0}}$}{magnetische Feldkonstante \nomunit{$\mu_{\mathrm{0}}=1,256\,637\,062\cdot10^{-6}\,\frac{\mathrm{Vs}}{\mathrm{Am}} \approx 4\pi\cdot10^{-7}\,\frac{\mathrm{Vs}}{\mathrm{Am}}$}}
\nomenclature[EW]{Wb}{Weber - magnetischer Fluss \nomunit{$\mathrm{Wb}=\mathrm{V}\cdot\mathrm{s}=\frac{\mathrm{kg}\cdot\mathrm{m}^2}{\mathrm{s}^2\cdot\mathrm{A}}$}}
\nomenclature[EM]{m}{Meter - Länge}
\nomenclature[EK]{kg}{Kilogramm - Masse}
\nomenclature[ES]{s}{Sekunde - Zeit}
\nomenclature[EN]{N}{Newton - Kraft \nomunit{$\mathrm{N}=\frac{\mathrm{kg}\cdot\mathrm{m}}{s^2}$}}
\nomenclature[EW]{W}{Watt - Wirkleistung \nomunit{$\mathrm{W}=\frac{\mathrm{J}}{\mathrm{s}}=\mathrm{A}\cdot\mathrm{V}=\frac{\mathrm{kg}\cdot\mathrm{m}^2}{\mathrm{s}^3}$}}
\nomenclature[EV]{V}{Volt - Spannung \nomunit{$\mathrm{V}=\frac{\mathrm{W}}{\mathrm{A}}=\frac{\mathrm{J}}{\mathrm{C}}=\frac{\mathrm{N}\cdot\mathrm{m}}{\mathrm{A}\cdot\mathrm{s}}=\frac{\mathrm{kg}\cdot\mathrm{m}^2}{\mathrm{s}^3\cdot\mathrm{A}}$}}
