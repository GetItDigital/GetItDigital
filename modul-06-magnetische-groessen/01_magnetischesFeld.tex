
\s{ 
	\section{Einleitung}
	Der Elektromagnetismus bildet die Grundlage vieler Technologien in unserem täglichen Leben.
	Die Magnetresonanztomographie ermöglicht zum Beispiel präzise Diagnosen und Therapien in der Medizin. 
	Die effiziente Erzeugung, Übertragung und Verteilung von elektrischer Energie wird durch Generatoren und Transformatoren gewährleistet. 
	In der Computertechnik sind magnetische Speicher, wie Festplattenlaufwerke, für die Datenverarbeitung und -speicherung unverzichtbar. 
	Die Grenzen der Anwendungsgebiete sind jedoch noch nicht erreicht. Zukünftige technologische Entwicklungen wie Quantencomputer, drahtlose Energieversorgung und magnetische Levitation werden diese weiter verschieben. 
	Ein fundiertes Verständnis magnetischer Größen ist entscheidend für die Weiterentwicklung und Optimierung existierender sowie zukünftiger Technologien.

	Dieses Kapitel bietet eine Einführung in die elektromagnetischen Wirkweisen und die dazugehörigen magnetischen Größen. 
	Zunächst wird das magnetische Feld beschrieben, gefolgt von einer Erklärung relevanter Größen für den magnetischen Kreis wie Durchflutung, magnetischer Fluss und magnetischer Widerstand.
	Anschließend werden die Funktionsweisen der Lorentzkraft, der Induktion und der Induktivität erläutert, gefolgt von der Berechnung des Energieinhalts im magnetischen Feld. Das Kapitel schließt mit einem Exkurs zum Skin-Effekt und Hall-Effekt.

	\begin{Lernziele}{Magnetische Größen}
		Die Studierenden
		\begin{itemize}
			\item kennen die grundlegenden magnetischen Größen im magnetischen Kreis.
			\item verstehen die physikalischen Wirkprinzipien hinter den einzelnen magnetischen Größen.
			\item können die Wechselwirkungen der magnetischen Größen zueinander beschreiben.
			\item können die einzelnen Größen im magnetischen Kreis berechnen.
		\end{itemize}
	\end{Lernziele}
}
\b{
	\begin{frame}{Lernziele}
		\begin{Lernziele}{Magnetische Größen}
			Die Studierenden
			\begin{itemize}
				\item kennen die grundlegenden magnetischen Größen im magnetischen Kreis.
				\item verstehen physikalischen Wirkprinzipien hinter den einzelnen magnetischen Größen.
				\item können die Wechselwirkungen der magnetischen Größen zueinander beschreiben.
				\item können die einzelnen Größen im magnetischen Kreis berechnen.
			\end{itemize}
		\end{Lernziele}
	\end{frame}
}

\begin{frame} \ftx{Magnetismus}
	\s{ 
		\section{Magnetismus}
		Der Magnetismus ist ein physikalisches Phänomen, das sich in Form von wechselwirkenden Kräften von magnetisierten bzw. magnetisierbaren Festkörpern und bewegten elektrischen Ladungen äußert. Die Kräfte werden mittels eines Magnetfeldes dargestellt. 
		Zu den geläufigsten Formen des Magnetismus werden der Elektromagnetismus und der Magnetismus von Festkörpern gezählt. Der Ferromagnetismus ist die bekannteste und wichtigste Art magnetisierter Festkörper und beschreibt das magnetische Verhalten einiger metallischer Körper, sogenannter ferromagnetischer Werkstoffe. \index{Ferromagnetismus} 
		Wie in Abbildung \ref{Magnet1} ersichtlich, besteht ein ferromagnetisches Material wie Eisen, Cobalt oder Nickel aus vielen kleinen Elementarmagneten\index{Elementarmagnet} in sogenannten Weissschen Bezirken\index{Weisssche Bezirke}, die durch Blochwände\index{Blochwände} voneinander getrennt sind. 
		In einem unmagnetisierten Körper sind diese Weissschen Bezirke willkürlich angeordnet und somit nicht ausgerichtet. Der Körper ist nach außen nicht (oder nur wenig) magnetisiert, da sich die Magnetisierungsrichtungen der einzelnen Bezirke großteils gegenseitig aufheben.
		
		Durch ein starkes externes Magnetfeld (die benötigte Stärke ist temperatur- und materialabhängig) können die ungeordneten Elementarmagnete parallel ausgerichtet werden. Durch die gleiche Ausrichtung der Weissschen Bezirke wird das Material selbst magnetisch.
	}
	\speech{Test}{1}{Das ist ein Test.}

	\fo{width=\textwidth}{width=\textwidth}{Aufbau_Permanentmagnet}{
		\put(24,0){\line(-1,1){2.8}}\put(24,0){Weisssche Bezirke}
		\put(30,3.5){\line(-1,1){4.4}}\put(30.3,3.5){Elementarmagnete}
		\put(36,7){\line(-1,1){3.8}}\put(36.3,7){Blochwände}
	}{{\bf Aufbau eines Permanentmagneten.} Links sind die  Weissschen Bezirke mit den Elementarmagneten willkürlich ausgerichtet, der Körper ist nicht magnetisiert. Der rechte Körper ist magnetisiert, da dort die Elementarmagnete der Weissschen Bezirke gleichgeordnet ausgerichtet sind.\label{Magnet1}}
\end{frame}

\subsection{Magnetisches Feld}
\begin{frame} \ftx{Magnetisches Feld}
	\s{
		Durch die geordnete Ausrichtung der Weissschen Bezirke entsteht ein außerhalb des Körpers messbares Magnetfeld. Magnetfelder sind Vektorfelder, die eine Krafteinwirkung auf magnetische Materialien im Raum ausüben. 
		Die Stärke des Magnetfeldes wird über die Größe der magnetischen Feldstärke $ \vec{H}$ beschrieben. Diese vektorielle Größe ordnet dabei jedem Punkt des Raums, auf den das Magnetfeld einwirkt, eine entsprechende Richtung mit einer spezifischen magnetischen Stärke ein. 
		Magnetfelder und die damit verbundene magnetische Feldstärke werden mittels Feldlinien dargestellt. 
		Magnetische Feldlinien besitzen charakteristische Eigenschaften:

		\begin{itemize}
			\item Magnetische Feldlinien sind immer geschlossen (Quellenfrei).
			\item Außerhalb des Magneten verlaufen sie vom Nord- zum Südpol.
			\item Sie treten immer senkrecht aus der Magnetoberfläche aus bzw. ein.
		\end{itemize}
		
		\fo{width=0.8\textwidth}{height=0.65\textheight}{Feldlinien_Permanentmagnet}{
			\put(75,60){\color{magnetfeld}\makebox[0pt]{N}}
			\put(75,7.5){\color{magnetfeld}\makebox[0pt]{S}}
		}{{\bf Feldlinien eines Permanentmagneten.} Links wurden die Feldlinien mit Hilfe von Metallspänen über einem Magnenten visualisert. Die rechte Grafik stellt die Eigenschaften magnetischer Feldlinien dar: Geschlossenheit, Verlauf außerhalb des Magneten von Nord nach Süd, senkrechter Austritt aus dem Magneten. \label{Magnet2}}
		\begin{Merksatz}{Magnetfeld} 
			Magnetfelder sind Vektorfelder, die an jedem Punkt des Raums, auf den sie einwirken, eine für den Punkt spezifische Richtung mit einer spezifischen Stärke besitzen. Die Stärke und Richtung werden mittels der magnetischen Feldstärke $ \vec{H}$ beschrieben.
		\end{Merksatz}
		
	}
	\b{
		\index{Magnetisches Feld}
		\onslide<1->{
			\fo{width=0.8\textwidth}{height=0.65\textheight}{Feldlinien_Permanentmagnet}{
				\put(75,60){\color{magnetfeld}\makebox[0pt]{N}}
				\put(75,7.5){\color{magnetfeld}\makebox[0pt]{S}}
			}{{\bf Feldlinien eines Permanentmagneten.} Links wurden die Feldlinien mit Hilfe von Metallspänen über einem Magnenten visualisert. Die rechte Grafik stellt die Eigenschaften magnetischer Feldlinien dar: Geschlossenheit, Verlauf außerhalb des Magneten von Nord nach Süd, senkrechter Austritt aus dem Magneten.}
			}
		\begin{itemize}
			\onslide<1->{
			\item Magnetische Feldlinien sind immer geschlossen (Quellenfrei).
			}
			\onslide<2->{
			\item Außerhalb des Magneten verlaufen sie vom Nord- zum Südpol.
			}
			\onslide<3->{
			\item Sie treten immer senkrecht aus der Magnetoberfläche aus bzw. ein.
			}
		\end{itemize}
	}
	
\end{frame}
