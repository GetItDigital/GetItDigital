\section{Energie im magnetischen Feld}
\begin{frame} \ftx{Energie im magnetischen Feld} 
	\s{Um die Energie, die in einer Spule gespeichert ist, zu berechnen, wird die in der Abbildung \ref{AbbSpule1} illustrierte Ringkernspule an eine Gleichspannungsquelle angeschlossen. Zum Zeitpunkt $t = 0$ wird die Spannungsquelle eingeschaltet. Die Spannung an der Spule ist dadurch zu jedem Zeitpunkt $t > 0$ gleich der Quellenspannung $U$. 
	Der durch die Spule geleitene Strom wird nach Gleichung \ref{GLInduktivitaet2} linear ansteigen. Die dadurch zugeführte Energie erzeugt eine Flussdichte im Spulenkern. 
	Um die Energie zu berechnen, die der Spule zugeführt wird, kann bekannte Gleichung der elektrischen Leistung \ref{GlArbeit} umgeformt werden und anschließend die Spannung durch Gleichung \ref{GLInduktivitaet2} ersetzt werden. 
	Als Ergebnis dieser Umformung erhält die Änderung der magnetischen Energie $\mathrm{d}W_{\mathrm{L}}$, welche aus der Multiplikation von Induktivität $L$, dem Strom durch die Induktität $i_{\mathrm{L}}$ und der Änderungsrate des Stroms durch die Induktität $\mathrm{d}i_{\mathrm{L}}$ errechnet (siehe Gleichung \ref{GlÄnderungmagnetischeEnergie}).
	
		
	% \f{width=0.5\textwidth}{width=0.5\textwidth}{Spule}{Ringkernspule mit einer angelegten Spannung \label{AbbSpule1}{\tiny(Quelle: \cite[Abbildung 6.26]{Albach})}}	

		\fo{width=0.35\textwidth}{width=0.35\textwidth}{Ringspule_MagFeld}{
			\put(6, 53){\color{current}$I$}
			\put(11, 1){\color{current}$I$}
			\put(76, 15){\color{magnetfeld}$\ell_{\mathrm{m}}$}
		}{{\bf Geschlossene Ringkernspule.} \label{AbbSpule1}}	
	}	
	\b{
		\begin{figure}[htbp]
			\centering
			\begin{minipage}{0.4\textwidth}
				\centering
				\fo{width=0.8\textwidth}{width=0.8\textwidth}{Ringspule_MagFeld}{
					\put(6, 53){\color{current}$I$}
					\put(11, 1){\color{current}$I$}
					\put(76, 15){\color{magnetfeld}$\ell_{\mathrm{m}}$}
				}{\bf Geschlossene Ringkernspule.}
			\end{minipage}
			\hfill
			\begin{minipage}{0.54\textwidth}
				\begin{eqa}
					\onslide<1->{p &= \frac{\d W}{\d t} = u\cdot i \qquad u = L\cdot\frac{\d i}{\d t}}\\
					\onslide<2->{\d W_{\mathrm{m}} &= u_{\mathrm{L}}\cdot i_{\mathrm{L}} \cdot \d t = L\cdot \frac{\d i}{\d t}\cdot i_{\mathrm{L}}\cdot \d t = L\cdot i_{\mathrm{L}} \cdot \d i_{\mathrm{L}}}
				\end{eqa}
				
				\begin{eqa}
				\begin{aligned}
					\onslide<3->{W_{\mathrm{m}} &= L\cdot \int\limits_0^I i_{\mathrm{L}}\cdot \mathrm{d}i_{\mathrm{L}} = \frac{1}{2}\cdot L\cdot I^2 \\
					&= \frac{1}{2}\cdot N \cdot \varPhi \cdot I}
				\end{aligned}
				\end{eqa}
			\end{minipage}
		\end{figure}
	}
	\s{
		\begin{eqa}				
				p &= \frac{\d W}{\d t} = u\cdot i \label{GlArbeit}\\
				u &= L\cdot\frac{\d i}{\d t} \tag{\ref{GLInduktivitaet2}}\\
				\d W_{\mathrm{m}}&=u_{\mathrm{L}}\cdot i_{\mathrm{L}} \cdot \d t = L\cdot \frac{\d i}{\d t}\cdot i_{\mathrm{L}}\cdot \d t = L\cdot i_{\mathrm{L}} \cdot \d i_{\mathrm{L}} \label{GlÄnderungmagnetischeEnergie}
		\end{eqa}

	Die Gesamtenergie ist das Integral über $\d W_{\mathrm{m}}$ vom Anfangswert $i_{\mathrm{L}}=0$ bis zum statischen Endwert $i_{\mathrm{L}} = I$. 
	Unter der Voraussetzung einer konstanten Induktivität $L$ gilt dann:
	
	\begin{equation}
	W_m = L \cdot \int_0^{I_L} i_L \cdot \d i_L = \frac{1}{2} \cdot L \cdot I^2 = \frac{1}{2} \cdot N \cdot \varPhi \cdot I
	\end{equation}}
	\nomenclature[FF]{$F$}{Kraft \nomunit{N}}
	\nomenclature[FP]{$P$}{Wirkleistung \nomunit{W}}
	\nomenclature[FW]{$W$}{Energie \nomunit{J}}

\end{frame}
\begin{frame} \ftx{Energie im magnetischen Feld} 
	\s{Die magnetische Energie kann auch aus den Feldgrößen durch Integration mit der magnetischen Flussdichte errechnet werden:
	\begin{equation}
		W_{\mathrm{m}} = \ell_{\mathrm{m}}\cdot A\cdot \int\limits_0^{B_L} H\d B = \ell_{\mathrm{m}}\cdot A \cdot \frac{B_{\mathrm{L}}^2}{2\cdot \mu_{\mathrm{r}}\cdot \mu_0} 
	\end{equation}}
	\b{Berechnung über die Feldgrößen:
	\begin{equation}
		\begin{aligned}
		\onslide<1->{W_{\mathrm{m}} &= \ell_{\mathrm{m}}\cdot A\cdot \int\limits_0^{B_L} H\d B \\
		  &= \ell_{\mathrm{m}}\cdot A \cdot \frac{B_{\mathrm{L}}^2}{2\cdot \mu_{\mathrm{r}}\cdot \mu_0}} \\[10pt]
		\onslide<2->{\frac{W_{\mathrm{m}}}{V} &= \frac{B_{\mathrm{L}}^2}{2\cdot \mu_0} = \frac{F}{A}}
		\end{aligned}
	\end{equation}}
	\pause
	\s{Bei einem Elektromagneten hat die Energiedichte im Luftspalt (also die Energie pro Volumen des Luftspalts) eine besondere Bedeutung: Sie steht für die \glqq Kraft des Elektromagneten\grqq. Die maximale Kraftwirkung befindet sich genau am Übergang vom Luftspalt zum Eisenkern -- entspricht also einem theoretisch unendlich schmalen Luftspalt. Da der Luftspalt, wie der Name schon sagt, meistens mit Luft gefüllt ist, ist im Normalfall $\mu_{\mathrm{r}} = 1$ und kann weggelassen werden.
	
	\begin{eq}
		\frac{W_{\mathrm{m}}}{V} = \frac{B_{\mathrm{L}}^2}{2\cdot \mu_0} = \frac{F}{A}
	\end{eq}
	}
\end{frame}